\documentclass[a4paper,12pt]{article}

\usepackage[polish]{babel}      
\usepackage[utf8]{inputenc}     
\usepackage[T1]{fontenc}        
\usepackage{lmodern}            

\usepackage{geometry}           
\geometry{
    a4paper,
    total={170mm,257mm},
    left=25mm,
    top=25mm,
}
\usepackage{graphicx}           
\usepackage{float}              
\usepackage{xcolor}             
\usepackage{titlesec}           
\usepackage{hyperref}           

\graphicspath{ {./screenshots/} }

\hypersetup{
    colorlinks=true,
    linkcolor=black,
    filecolor=magenta,      
    urlcolor=blue,
    pdftitle={Instrukcja GitHub Desktop},
}

\titleformat{\section}
{\normalfont\Large\bfseries\color{darkgray}}{\thesection}{1em}{}

\title{
    \vspace{2cm}
    \Huge \textbf{Instrukcja Obsługi}\\
    \LARGE Aplikacja GitHub Desktop
    \vspace{1cm}
}
\author{\textit{Dział IT / Wsparcie Techniczne}}
\date{\today}

\begin{document}

\maketitle
\thispagestyle{empty} 

\vfill
\begin{center}
    \large Dokumentacja techniczna v1.0
\end{center}
\newpage

\tableofcontents
\newpage

\section{Wstęp}
GitHub Desktop to aplikacja z graficznym interfejsem użytkownika (GUI), która ułatwia pracę z systemem kontroli wersji Git. Niniejsza instrukcja krok po kroku przeprowadzi Cię przez kluczowe funkcjonalności programu, pozwalając na efektywne zarządzanie kodem bez konieczności używania wiersza poleceń.

\section{Klonowanie Repozytorium}
Aby rozpocząć pracę nad istniejącym projektem, należy go najpierw sklonować (pobrać) na dysk lokalny.

\begin{enumerate}
    \item Uruchom aplikację GitHub Desktop.
    \item Kliknij w menu \textbf{File} i wybierz opcję \textbf{Clone repository...} (lub użyj skrótu \texttt{Ctrl+Shift+O}).
    \item W otwartym oknie wybierz zakładkę \textbf{GitHub.com}, znajdź swoje repozytorium na liście i zaznacz je.
    \item Wskaż ścieżkę lokalną (Local path), gdzie pliki mają zostać zapisane.
    \item Kliknij przycisk \textbf{Clone}.
\end{enumerate}

\begin{figure}[H]
    \centering
    \includegraphics[width=0.8\textwidth]{clone.png}
    \caption{Okno klonowania repozytorium w GitHub Desktop.}
    \label{fig:clone}
\end{figure}

\section{Praca z Gałęziami (Branching)}
Dobrą praktyką jest praca na osobnych gałęziach (branch) dla każdej nowej funkcjonalności, aby nie modyfikować bezpośrednio gałęzi głównej (main/master).

\begin{enumerate}
    \item Kliknij przycisk \textbf{Current Branch} na górnym pasku narzędzi.
    \item Wpisz nazwę nowej gałęzi w polu tekstowym (np. \texttt{feature-logowanie}).
    \item Kliknij przycisk \textbf{New Branch}.
    \item Jeśli aplikacja zapyta, czy przenieść obecne zmiany, wybierz \textit{Bring my changes...} lub \textit{Leave my changes...} zależnie od potrzeby.
\end{enumerate}

\begin{figure}[H]
    \centering
    \includegraphics[width=0.8\textwidth]{branch.png}
    \caption{Tworzenie nowej gałęzi.}
    \label{fig:branch}
\end{figure}

\section{Tworzenie Commitów}
Po wprowadzeniu zmian w plikach (edycja kodu, dodanie grafik), należy je zatwierdzić (zrobić commit).

\begin{enumerate}
    \item Po lewej stronie w panelu \textbf{Changes} zobaczysz listę zmodyfikowanych plików.
    \item Zaznacz pliki, które chcesz dołączyć do commita (domyślnie zaznaczone są wszystkie).
    \item W polu \textbf{Summary} (wymagane) wpisz krótki tytuł zmiany.
    \item W polu \textbf{Description} (opcjonalne) możesz dodać szerszy opis.
    \item Kliknij niebieski przycisk \textbf{Commit to <nazwa-gałęzi>}.
\end{enumerate}

\begin{figure}[H]
    \centering
    \includegraphics[width=0.8\textwidth]{commit.png}
    \caption{Panel zatwierdzania zmian (Commit).}
    \label{fig:commit}
\end{figure}

\section{Synchronizacja (Push/Pull)}
Zmiany zatwierdzone lokalnie (commit) muszą zostać wysłane na serwer (Push), aby inni mogli je zobaczyć. Należy również pobierać zmiany innych (Pull).

\subsection{Wysyłanie zmian (Push)}
Jeśli masz nowe commity lokalnie, na górnym pasku pojawi się przycisk \textbf{Push origin}. Kliknij go, aby wysłać zmiany na GitHub.

\subsection{Pobieranie zmian (Fetch/Pull)}
Aby sprawdzić, czy są nowe zmiany na serwerze, kliknij \textbf{Fetch origin}. Jeśli pojawią się nowe zmiany, przycisk zmieni się na \textbf{Pull origin}. Kliknij go, aby zaktualizować lokalne pliki.

\begin{figure}[H]
    \centering
    \includegraphics[width=0.8\textwidth]{push.png}
    \caption{Przycisk synchronizacji (Push/Fetch) na górnym pasku.}
    \label{fig:push}
\end{figure}

\section{Tworzenie Pull Request (PR)}
Gdy skończysz pracę na swojej gałęzi i wyślesz ją na serwer (Push), możesz utworzyć prośbę o włączenie zmian (Pull Request).

\begin{enumerate}
    \item Upewnij się, że Twoja gałąź jest opublikowana na serwerze.
    \item Kliknij przycisk \textbf{Create Pull Request} (znajdziesz go na głównym ekranie po zrobieniu Push lub w menu \textbf{Branch}).
    \item Aplikacja otworzy przeglądarkę internetową ze stroną GitHub.
    \item Uzupełnij opis PR i kliknij \textbf{Create pull request} na stronie www.
\end{enumerate}

\begin{figure}[H]
    \centering
    \includegraphics[width=0.8\textwidth]{pr.png}
    \caption{Przycisk inicjujący tworzenie Pull Request.}
    \label{fig:pr}
\end{figure}

\newpage
\section{Podsumowanie}
Powyższa instrukcja pokrywa podstawowy cykl pracy developera: pobranie kodu, stworzenie gałęzi, zapisanie zmian, synchronizację oraz zgłoszenie zmian do weryfikacji. Regularne wykonywanie operacji \textbf{Fetch} oraz częste \textbf{Commity} to klucz do uniknięcia konfliktów w kodzie.

\end{document}
